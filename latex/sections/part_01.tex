\documentclass[../main.tex]{subfiles}

\graphicspath{{\subfix{../images/}}}

\begin{document}

\section{Part 1 - Modelling and Simulations}

\subsection{Task 1.1}

Draw the equivalent circuit model to be used to simulate the crosstalk in each of the 3 cases.

\vspace{10pt}
The model must include the driver circuit (linear model), package parasitics (simple model), transmission lines, and probe points, and receiver circuit (linear model).

\solution

To model the driver, receiver and all package parasitics we will use an IBIS model for the 74ALVC244 buffer chip. We start by describing how the IBIS specification can be used to model the buffer. 

\subsubsection{Buffer Output Model}

Consider the generic three-state output buffer shown in Figure \ref{fig:ibis-buffer}.

\begin{figure}[h]
    \centering
    \includegraphics[width=0.5\textwidth]{three-state-ibis.png}
    \caption{Three-state output buffer\footcite{ibis}.}
    \label{fig:ibis-buffer}
\end{figure}

In our case we will model the output of the CMOS devices as a voltage source with a series resistance. The capacitor $C_{comp}$ will be used to model the ramp rate of the falling and rising waveforms. $C_{comp}$ is the silicon die capacitance, and does \textbf{not} include the package parasitics. $R_{pkg}, C_{pkg}, L_{pkg}$ are the electrical characteristics for each pin-to-buffer connection. They are the lumped elements of the overall package.

\vspace{10pt}
The linear model of the driver circuit is shown in Figure \ref{fig:driver-package}. $V_1, R_1. C_1$ constitute the driver model, and $R_2, C_2, L_2$ are the QFN package parasitics.

\begin{figure}[h]
    \centering
    \fbox{\includegraphics[width=0.8\textwidth]{ltspice_driver.png}}
    \caption{74ALVC244 driver and package model (QFN).}
    \label{fig:driver-package}
\end{figure}

\newpage

\subsubsection{Buffer Input Model}

Consider the generic input buffer shown in Figure \ref{fig:ibis-input}. Again we have $C_{pkg}, R_{pkg}, L_{pkg}$ to model the package parasitics. Since the signal will drive the gate(s) of a CMOS circuit, it can be modelled as a single capacitor $C_{comp}$.

\begin{figure}[h]
    \centering
    \includegraphics[width=0.5\textwidth]{input-ibis.png}
    \caption{Input buffer\footcite{ibis}.}
    \label{fig:ibis-input}
\end{figure}

The linear model of the receiver circuit is shown in Figure \ref{fig:receiver-package}. $C_4$ describes the receiver and $C_3, L_2, R_7$ model the package.

\begin{figure}[h]
    \centering
    \fbox{\includegraphics[width=0.8\textwidth]{ltspice_receiver.png}}
    \caption{74ALVC244 receiver and package model (QFN).}
    \label{fig:receiver-package}
\end{figure}

\subsubsection{Transmission Line Model}

To model the transmission lines we will use a \textit{lossless lumped element model}, essentially a series of resistors, capacitors, and inductors. The minimum number of segments to accurately model a transmission line is given by (\ref{eq:seg_min}).

\begin{equation} \label{eq:seg_min}
    N_{seg, min} = 10 \frac{TD}{T_{r}}
\end{equation}

Where $TD$ is the delay of the transmission line, and $T_{r}$ is the rise time of the signal. The time delay, TD, is given by (\ref{eq:td}).

\begin{equation} \label{eq:td}
    TD = \sqrt{L_{total} * C_{total}}
\end{equation}

Denoting the transmission line length $x$, we have the following values for all 3 cases:

\begin{align*}
    x &= 0.01\,\si{m} \\
    T_r &= 500\,\si{ps}
\end{align*}

\newpage

\begin{table}[t]
    \centering
    \begin{tabular}{l|l l}
        \toprule[1pt]
        \textbf{Case} & \textbf{Inductance Matrix} & \textbf{Capacitance Matrix} \\
        \midrule
        Side-by-side microstrip &
        $L_{mat,1} =
        \begin{bmatrix}
            283.4 & 52.2 \\
            52.2 & 283.4
        \end{bmatrix}$ &
        $C_{mat,1} =
        \begin{bmatrix}
            128.5 & -12.0 \\
            -12.0 & 128.5
        \end{bmatrix}$ \\
        && \\
        Side-by-side stripline &
        $L_{mat,2} =
        \begin{bmatrix}
            318.5 & 45.3 \\
            45.3 & 318.5
        \end{bmatrix}$ &
        $C_{mat,2} =
        \begin{bmatrix}
            153.3 & -21.8 \\
            -21.8 & 153.3
        \end{bmatrix}$ \\
        && \\
        Broadside stripline &
        $L_{mat,3} =
        \begin{bmatrix}
            318.0 & 101.4 \\
            101.4 & 318.0
        \end{bmatrix}$ &
        $C_{mat,3} =
        \begin{bmatrix}
            167.5 & -53.5 \\
            -53.5 &167.53
        \end{bmatrix}$ \\
        \bottomrule[1pt]
    \end{tabular}
    \caption{Inductance and capacitance matrices for the three transmission line cases.}
    \label{tab:lc-matrix}
\end{table}

The inductor and capacitor matrices for all three cases are shown in Table \ref{tab:lc-matrix}. The time delay, TD, and thereafter the minimum number of segments, $N_{seg, min}$, are calculated for each case as follows:

\begin{equation*}
    TD_1 = x \sqrt{L_{self} C_{self}} = 0.01\,\si{m} \sqrt{283.4\,\si{nH/m} * 128.5\,\si{pF/m}} = 60.3\,\si{ps}
\end{equation*}
\begin{equation*}
    TD_2 = x \sqrt{L_{self} C_{self}} = 0.01\,\si{m} \sqrt{318.5\,\si{nH/m} * 153.3\,\si{pF/m}} = 69.9\,\si{ps}
\end{equation*}
\begin{equation*}
    TD_3 = x \sqrt{L_{self} C_{self}} = 0.01\,\si{m} \sqrt{318.0\,\si{nH/m} * 167.5\,\si{pF/m}} = 73.0\,\si{ps}
\end{equation*}
\begin{equation*}
    N_{seg,1} = 10 \frac{TD_1}{T_r} = 10 \frac{60.3\,\si{ps}}{500\,\si{ps}} = 1.21 \approx 2
\end{equation*}
\begin{equation*}
    N_{seg,2} = 10 \frac{TD_2}{T_r} = 10 \frac{69.9\,\si{ps}}{500\,\si{ps}} = 1.40 \approx 2
\end{equation*}
\begin{equation*}
    N_{seg,3} = 10 \frac{TD_3}{T_r} = 10 \frac{73.0\,\si{ps}}{500\,\si{ps}} = 1.46 \approx 2
\end{equation*}

We see that all cases \textit{luckily} only require 2 segments. The lumped element model for the transmission lines is shown in Figure \ref{fig:transmission-line}. Inductors $L_{11}, L_{21}$ and $L_{12}, L_{22}$ are coupled, and capacitors $C_{m}, C_{m2}$ are the mutual capacitances. Together they model the crosstalk between the lines. Resistors $R_3, R_4, R_5, R_6, R_8$ are physical SMD resistors on the PCB.

\begin{figure}[h]
    \centering
    \fbox{\includegraphics[width=0.8\textwidth]{ltspice_tl.png}}
    \caption{Lumped element model for the transmission lines.}
    \label{fig:transmission-line}
\end{figure}

\subsubsection{Combined Model}

% INSERT COMBINED MODEL HERE
% REMEMBER PROBE POINTS

\newpage

\subsection{Task 1.2}

Find and estimate the values of the different parameters based on data from the HW test-board, PCB schematic, and the datasheet and IBIS model for the driver/receiver IC. Assume that the aggressor line is matched at the driver end.

\solution

\subsubsection{Output Buffer Model Parameters}

Capacitor $C_{comp}$ is found to be $6.04\,\si{pF}$ (typically) in the IBIS model for the 74ALVC244 at a supply voltage of $V_{DD} = 3.3\,\si{V}$. Package parameters depend on which output pins are used and are summarized in Table \ref{tab:pkg-params}.

\begin{table}[h]
    \centering
    \begin{tabular}{l|l l l l}
        \toprule[1pt]
        \textbf{Case} & \textbf{Pin} & $R_{pkg}$ [$\si{\ohm}$] & $C_{pkg}$ [$\si{pF}$] & $L_{pkg}$ [$\si{nH}$] \\
        \midrule
        Side-by-side microstrip & 1Y2 & 0.09336 & 0.15918 & 1.33430 \\
        Side-by-side stripline  & 1Y3 & 0.10013 & 0.14031 & 1.44133 \\
        Broadside stripline     & 1Y4 & 0.11819 & 0.15918 & 1.73153 \\
        \bottomrule[1pt]
    \end{tabular}
    \caption{Package parameters for the 74ALVC244 buffer outputs (QFN).}
    \label{tab:pkg-params}
\end{table}

It is assumed that $R_1 = 10.9\,\si{\Omega}$, even though it varies depending on if it "pulls" up or down.

\subsubsection{Input Buffer Model Parameters}

Capacitor $C_{comp}$ is found to be $4.21\si{pF}$ (typically) in the IBIS model. Package parameters are summarized in Table \ref{tab:pkg-params2}.

\begin{table}[h]
    \centering
    \begin{tabular}{l|l l l l}
        \toprule[1pt]
        \textbf{Case} & \textbf{Pin} & $R_{pkg}$ [$\si{\ohm}$] & $C_{pkg}$ [$\si{pF}$] & $L_{pkg}$ [$\si{nH}$] \\
        \midrule
        Side-by-side microstrip & 1A2 & 0.09598 & 0.13711 & 1.38443 \\
        Side-by-side stripline  & 1A3 & 0.09315 & 0.12663 & 1.34060 \\
        Broadside stripline     & 1A4 & 0.10195 & 0.14411 & 1.47927 \\
        \bottomrule[1pt]
    \end{tabular}
    \caption{Package parameters for the 74ALVC244 buffer inputs (QFN).}
    \label{tab:pkg-params2}
\end{table}

\subsubsection{Transmission Line Model Parameters}

To model the transmission line we must the values for the inductors and capacitors in the lumped element model. We also need the inductive coupling coefficient $K$. Denoting the entries in the L, C matrices as $C_{i,j}, L_{i,j}$, the lumped element model values are given by (\ref{eq:cseg_value}), (\ref{eq:cmseg_value}), and (\ref{eq:lseg_value}).

\begin{flalign} \label{eq:cseg_value}
    &C_{seg} = \frac{x}{N_{seg}} \left( C_{11} + C_{12} \right)&
\end{flalign}
\begin{flalign} \label{eq:cmseg_value}
    &C_{m, seg} = \frac{x}{N_{seg}} (-C_{12})&
\end{flalign}
\begin{flalign} \label{eq:lseg_value}
    &L_{seg} = \frac{x}{N_{seg}} L_{11}&
\end{flalign}
\begin{flalign} \label{eq:lseg_value}
    &K = \frac{L_{12}}{\sqrt{L_{11}L_{22}}}&
\end{flalign}

The values for the lumped element model are summarized in Table \ref{tab:tl-params}.

\newpage

\begin{table}[h]
    \centering
    \begin{tabular}{l|l l l l}
        \toprule[1pt]
        \textbf{Case} & $K$ & $C$ [$\si{pF}$] & $C_m$ [$\si{pF}$] & $L$ [$\si{nH}$] \\
        \midrule
        Side-by-side microstrip & 0.184 & 0.583 & 0.060 & 1.42 \\
        Side-by-side stripline  & 0.142 & 0.657 & 0.109 & 1.59 \\
        Broadside stripline     & 0.319 & 0.570 & 0.268 & 1.59 \\
        \bottomrule[1pt]
    \end{tabular}
    \caption{Transmission line lumped element model values.}
    \label{tab:tl-params}
\end{table}


\newpage

\subsection{Task 1.3}

For each of the transmission lines calculate the equivalent circuit model of the system of the coupled transmission lines. Assume $T_{rise} = 500\,\si{ps}$.

\solution

\subsection{Task 1.4}

From the datasheet - what are the minimum high input and output levels and the maximum low input and output levels. (Assume $V_{CC} = 3.3\,\si{V} \pm 5\%$, and use the same max current load $\pm 12 \,\si{mA}$ for both output low and high).

\solution

\subsection{Task 1.5}

Simulate for each of the transmission line configurations the circuit model (Driver 0-3.3V, $T_{rise} = 500\,\si{ps}$).

\vspace{10pt}
Document the voltage curves over time, at the driver and receiver ends of the aggressor line, and the near-end (NEXT) and far-end (FEXT) crosstalk at the victim line.

\vspace{10pt}
Determine the critical crosstalk level in the low state and (high) state. The critical crosstalk level when the line is in a low (high) state is the maximum (minimum) voltage over the victim line due to crosstalk.

\solution

\subsection{Task 1.6}


Determine crosstalk from the simple quantitative formulas in the textbook (remark – both the aggressor line and the victim line are not matched at the receiver ends:

\solution

The simple quantitative formula for the crosstalk voltage at the victim line due to the aggressor line is given by:

\begin{equation*}
    V_x = V_{crosstalk} (1+ \frac{R-Z_o}{R+Z_o}) 
\end{equation*}

Where $V_x$ is the crosstalk at one end of the victim line with the adjustment for the non-ideal termination, $R$ is the impedance of the termination, $Z_o$ is the characteristic impedance of the transmission line and $V_{crosstalk}$ is the crosstalk voltage assuming ideal termination.

\vspace{10pt}

The equation for the ideal $V_{crosstalk}$ is dependent on the termination style of the termination, with the possibilities being double terminated, near-end terminated, and far-end terminated. In this case the victim line is double terminated leading to the crosstalk voltage being given by:

\begin{equation*}
    V_{near} = V_{crosstalk} \left(1+ \frac{R-Z_o}{R+Z_o} \right) = \left(\frac{V(input)}{4} \left(\frac{L_M}{L} + \frac{C_M}{C} \right) \right) \left(1+ \frac{R-Z_o}{R+Z_o} \right)
\end{equation*}

\begin{equation*}
    V_{far} = V_{crosstalk} \left(1+ \frac{R-Z_o}{R+Z_o} \right) = \left(-\frac{V(input)X\sqrt{LC}}{2Tr} \left(\frac{L_M}{L} - \frac{C_M}{C} \right) \right) \left(1+ \frac{R-Z_o}{R+Z_o} \right)
\end{equation*}

Where $V(input)$ is the input to the aggressor line, $X$ is the length of the lines, $Tr$ is the risetime of the signal, $L_M$, $L$, $C_M$ and $C$ being the mutual and self inductance and capacitance respectively.

\vspace{10pt}
Most of these values are given or stated, the rest needs to be calculated. The value needed for the crosstalk formula that is uknown is the characteristic impedance $Z_0$. To determine this the formulas usually depend on whether it's a microstrip, symmetrical stripline or offset stripline but since the self- and mutual inductance and capacitance are given, we can use this simpler formula instead:

\begin{equation*}
    Z_0 = \sqrt{\frac{L}{C}}
\end{equation*}

\vspace{10pt}

So, the characteristic impedance of the different transmission lines (microstrip, side-by-side stripline, broadside stripline) are:

\begin{equation}
    Z_{mic} = \sqrt{\frac{L_{mic}}{C_{mic}}} = \sqrt{\frac{283.4}{128.5}} \cong 1.485
\label{eq:z_mic}
\end{equation}

\begin{equation}
    Z_{ss} = \sqrt{\frac{L_{ss}}{C_{ss}}} = \sqrt{\frac{318.5}{153.3}} \cong 1.441
\label{eq:z_ss}
\end{equation}

\begin{equation}
    Z_{bs} = \sqrt{\frac{L_{bs}}{C_{bs}}} = \sqrt{\frac{318}{167.5}} \cong 1.378
\label{eq:z_bs}
\end{equation}

Now that the characteristic impedance have been found, we can use the equation for the crosstalk. Firstly we must find the values needed for it. These are gathered in the table \ref{tab:cross_talk_table}.

\begin{table}[H]
\resizebox{\textwidth}{!}{%
\begin{tabular}{lll}
\textbf{Variable} & \textbf{Description}                      & \textbf{Value}  \\\hline
\multicolumn{1}{c}{R}           & Termination resistor                                  & 1 k$\ohm$ \\
\multicolumn{1}{c}{$Z_0$}       & Characteristic Impedence for the transmission line    & See equations \ref{eq:z_mic} - \ref{eq:z_bs}\\
\multicolumn{1}{c}{$V(input)$}  & Input voltage of the aggressor line                   & 3.3V \\
\multicolumn{1}{c}{L}           & Self inductance of the victim line                    & See table \ref{tab:lc-matrix} \\
\multicolumn{1}{c}{$L_M$}       & Mutual inductance of the lines                        & See table \ref{tab:lc-matrix} \\
\multicolumn{1}{c}{C}           & Self capacitance of the victim line                   & See table \ref{tab:lc-matrix} \\
\multicolumn{1}{c}{$C_M$}       & Mutual capacitance of the lines                       & See table \ref{tab:lc-matrix} \\
\multicolumn{1}{c}{X}           & Length of the transmission line                       & 10 cm \\
\multicolumn{1}{c}{Tr}          & Rise time of the aggresor line pulses                 & 500 ps  
\end{tabular}%
}
\caption{Table over needed values for calculating the crosstalk}
\label{tab:cross_talk_table}
\end{table}

Using these values to calculate the crosstalk at the far end of the microstrips would give the following:

\begin{multline}
    V_{far} = \left(-\frac{V(input)X\sqrt{LC}}{2Tr} \left(\frac{L_M}{L} - \frac{C_M}{C} \right) \right) \left(1+ \frac{R-Z_{mic}}{R+Z_{mic}} \right) \\
    = \left(-\frac{3.3V \cdot 0.01m\sqrt{283.4 nH \cdot 128.5 pF}}{2 \cdot 500 ps} \left(\frac{52.2 nH}{283.4 nH} - \frac{-12 pF}{128.5 pF} \right) \right) \left(1+ \frac{1k\ohm - 1.485\ohm}{1k\ohm+1.485\ohm} \right) \\ 
  = \left(-\frac{33mmV\sqrt{36417 nHpF}}{1000 ps} \left(0.184 - (-0.093) \right) \right) \left(1+ 0.91 \right) \\
  \cong -0.199V  \cdot 0.277 \cdot 1.91 \\ \cong -0.1056 V
\label{eq:FEXT_microstrip}
\end{multline}

So there should be roughly -105mV at the far end of the microstrip caused by the crosstalk from the aggressorline. This is then used to find the low- and high-state steady-state and critical level of voltage at the near- and far-end of the transmission line. The crosstalk for other scenarios; microstrip, side-by-side striplines and broadside, either far or near-end, can be found in a similar manner and the results are summarized in table \ref{tab:crosstalk_results}. 

\vspacec{10pt}

In the theoretical model, the steady state level is the voltage that the victim line would have if there was no crosstalk. This is simply the voltage applied to the victim line; 0V for low state and 3.3V for high state.
%Jeg ved squ ikke lige om det her er rigtigt, jeg skal lige dobbelt tjekke med Hooman.

\vspacec{10pt}

The critical level is the voltage achieved due to the crosstalk that pushes the signal closest to the undefined terriory between low and high. Since the crosstalk appears both positively and negatively due to it being pulses, we find the critical level by the worst case scenario, either adding or subtracting the crosstalk voltage from the victim line. For example the critical level of the low state at the far end of the microstrip would be:

\begin{equation}
    Low\,State\,Critical\,Level = Steady\,State\,Level + |V_{far}| = 0 V + |-0.1056 V| = 0.1056 V
\end{equation}

If instead it was the critical level during high state, we would have:

\begin{equation}
    High\,State\,Critical\,Level = Steady\,State\,Level - |V_{far}| = 3.3 V - |-0.1056 V| = 3.1944 V
\end{equation}

The steady state and critical levels for all scenarios are summarized in table \ref{tab:crosstalk_results}.

\begin{table}[H]
    \centering
    \begin{tabular}{cl|ll|ll|ll}
    \multicolumn{2}{c|}{\textbf{Setting}}                                   & \multicolumn{2}{c|}{\textbf{Microstrip}} & \multicolumn{2}{l|}{\textbf{\textbf{Side-by-Side}}} & \multicolumn{2}{c}{\textbf{Broadside}}  \\ \hline
    \textbf{End}                       &                                    & \textit{N} & \textit{F}                  & \textit{N}     & \textit{F}                         & \textit{N}               & \textit{F}   \\ \hline
    \textbf{Steady State}              &                                    & 0          & 0                           & 3.3            & 3.3                                & 0                        & 3.3          \\
    \textbf{Crosstalk}                 &                                    & 0.143      & -0.106                      & $3.846 \cdot 10^{-5}$ & -0.125                      & $-8.46\cdot 10^{-4}$     & -0.295       \\
    \multirow{2}{*}{\textbf{Critical}} & \multicolumn{1}{c|}{\textit{Low}}  & 0.143      & 0.106                       & $3.846 \cdot 10^{-5}$ & 0.125                       & $8.46\cdot 10^{-4}$     & 0.295         \\
                                       & \multicolumn{1}{c|}{\textit{High}} & 3.157      & 3.194                       & 3.3            & 3.175                              & 3.299                    & 3.005                  
    \end{tabular}
    \caption{Table over Steady State, Crosstalk and Critical level voltage at respectively Near-end and Far-end at Microstrips, Side-by-Side Striplines and Broadside Striplines. All values are in volts.}
\label{tab:crosstalk_results}
\end{table}

\textbf{Does this match the simulation results in task 1.5??}

 
\vspace{10pt}

Noter fra lab session : Spørg Hooman

Comment on the importance of the indivual parameters in the crosstalk formula.

\vspace{10pt}

Comment on things that can impact the difference between the analytical, simulated and measured crosstalk.
- noise
- cable losses
- not a lossless transmission line


\end{document}
