\documentclass[../main.tex]{subfiles}

\graphicspath{{\subfix{../images/}}}

\begin{document}

\section{Part 1 - Modelling and Simulations}

\subsection{Task 1.1}

Draw the equivalent circuit model to be used tol simulate the crosstalk in each of the 3 cases.

\vspace{10pt}
The model must include the driver circuit (linear model), package parasitics (simple model), transmission lines, and probe points, and receiver circuit (linear model).

\solution

\subsection{Task 1.2}

Find and estimate the values of the different parameters based on data from the HW test-board, PCB schematic, and the datasheet and IBIS model for the driver/receiver IC. Assume that the aggressor line is matched at the driver end.

\solution

\subsection{Task 1.3}

For each of the transmission lines calculate the equivalent circuit model of the system of the coupled transmission lines. Assume $T_{rise} = 500\,\si{ps}$.

\solution

\subsection{Task 1.4}

From the datasheet - what are the minimum high input and output levels and the maximum low input and output levels. (Assume $V_{CC} = 3.3\,\si{V} \pm 5\%$, and use the same max current load $\pm 12 \,\si{mA}$ for both output low and high).

\solution

\subsection{Task 1.5}

Simulate for each of the transmission line configurations the circuit model (Driver 0-3.3V, $T_{rise} = 500\,\si{ps}$).

\vspace{10pt}
Document the voltage curves over time, at the driver and receiver ends of the aggressor line, and the near-end (NEXT) and far-end (FEXT) crosstalk at the victim line.

\vspace{10pt}
Determine the critical crosstalk level in the low state and (high) state. The critical crosstalk level when the line is in a low (high) state is the maximum (minimum) voltage over the victim line due to crosstalk.

\solution

\end{document}
